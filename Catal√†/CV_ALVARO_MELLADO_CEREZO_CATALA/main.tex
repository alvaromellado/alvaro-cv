%%%%%%%%%%%%%%%%%%%%%%%%%%%%%%%%%%%%%%%%%
% Developer CV
% LaTeX Templateq
% Version 1.0 (28/1/19)
%
% This template originates from:
% http://www.LaTeXTemplates.com
%
% Authors:
% Jan Vorisek (jan@vorisek.me)
% Based on a template by Jan Küster (info@jankuester.com)
% Modified for LaTeX Templates by Vel (vel@LaTeXTemplates.com)
%
% License:
% The MIT License (see included LICENSE file)
%
%%%%%%%%%%%%%%%%%%%%%%%%%%%%%%%%%%%%%%%%%

%----------------------------------------------------------------------------------------
%	PACKAGES AND OTHER DOCUMENT CONFIGURATIONS
%----------------------------------------------------------------------------------------

\documentclass[9pt]{developercv} % Default font size, values from 8-12pt are recommended

%----------------------------------------------------------------------------------------

\begin{document}

%----------------------------------------------------------------------------------------
%	TITLE AND CONTACT INFORMATION
%----------------------------------------------------------------------------------------

\begin{minipage}[t]{0.6\textwidth} % 45% of the page width for name
	\vspace{-\baselineskip} % Required for vertically aligning minipages

	% If your name is very short, use just one of the lines below
	% If your name is very long, reduce the font size or make the minipage wider and reduce the others proportionately
	\colorbox{black}{{\HUGE\textcolor{white}{\textbf{\MakeUppercase{Álvaro}}}}} % First name

	\colorbox{black}{{\HUGE\textcolor{white}{\textbf{\MakeUppercase{Mellado}}}}} % Last name

	\vspace{6pt}

	{\huge Enginyeria de Sistemes TIC} % Career or current job title
\end{minipage}
\begin{minipage}[t]{0.275\textwidth} % 27.5% of the page width for the first row of icons
	\vspace{-\baselineskip} % Required for vertically aligning minipages

	% The first parameter is the FontAwesome icon name, the second is the box size and the third is the text
	% Other icons can be found by referring to fontawesome.pdf (supplied with the template) and using the word after \fa in the command for the icon you want
	\icon{MapMarker}{12}{Manresa}\\
	\icon{Phone}{12}{+34 620 395 887}\\
	\icon{At}{12}{\href{mailto:alvaro@mellado.cat}{alvaro@mellado.cat}}\\
\end{minipage}
\begin{minipage}[t]{0.275\textwidth} % 27.5% of the page width for the second row of icons
	\vspace{-\baselineskip} % Required for vertically aligning minipages

	% The first parameter is the FontAwesome icon name, the second is the box size and the third is the text
	% Other icons can be found by referring to fontawesome.pdf (supplied with the template) and using the word after \fa in the command for the icon you want

\end{minipage}

\vspace{0.5cm}

%----------------------------------------------------------------------------------------
%	INTRODUCTION, SKILLS AND TECHNOLOGIES
%----------------------------------------------------------------------------------------

\cvsect{Quí Sóc?}

\begin{minipage}[t]{0.4\textwidth} % 40% of the page width for the introduction text
	\vspace{-\baselineskip} % Required for vertically aligning minipages
	Sóc un enginyer de sistemes
	TIC, interessat per la sobirania tecnològica, la
	sostenibilitat dels serveis TIC i les tecnologies
	Web.
	\\
	\\
	Més enllà del que m’ha ensenyat el meu
	grau he adquirit molta experiència en coor-
	dinació d’equips en diferents àmbits tot i que
	aquests no han sigut laborals.
\end{minipage}
\hfill % Whitespace between
\begin{minipage}[t]{0.5\textwidth} % 50% of the page for the skills bar chart
	\vspace{-\baselineskip} % Required for vertically aligning minipages
	\begin{barchart}{5.5}
		\baritem{Rust}{60}
		\baritem{Python}{100}
		\baritem{C}{80}
		\baritem{GIT}{100}
		\baritem{C\#}{80}
	\end{barchart}
\end{minipage}

%----------------------------------------------------------------------------------------
%	EXPERIENCE
%----------------------------------------------------------------------------------------

\cvsect{Experiéncia}

\begin{entrylist}
	\entry
	{2023 -- Actualitat\\}
	{Software Enginyer}
	{Opportunities Ground SL}
	{Empresa especialitzada en el món dels Recursos Humans i la selecció de personal. Ens encarreguem d'inferir tant currículums com ofertes de treball per poder fer un matching entre candidats i empreses. Estalviant temps en el procés de selecció.
		\\ \texttt{Python}\slashsep\texttt{Linux}\slashsep\texttt{CircleCI}\slashsep\texttt{Docker}\slashsep\texttt{Flask}\slashsep\texttt{FastAPI}\slashsep\texttt{GitHub Actions}}
	\entry
	{2024 -- Actualitat\\}
	{Professor Associat}
	{Universitat Politècnica de Catalunya}
	{Professor associat en l'assignatura d'Informàtica a l'EPSEM (Escola Politècnica Superior d'Enginyeria de Manresa) de la UPC. L'assignatura és de primer curs i se centra en la programació en Python i l'ús de Linux.
		\\ \texttt{Python}\slashsep\texttt{Linux}}
	\entry
	{2022 -- 2023\\}
	{Desenvolupador backend / DevOps / IT}
	{IZI Record}
	{Empresa emergent del sector multimèdia que proporciona un servei d'edició automàtica de vídeos.

		Empresa on m'he dedicat a mantenir, gestionar i renovar tot el programari no relacionat amb Intel·ligència Artificial  L'eina principal ha sigut Flask, però també he treballat a renovar
		l'arquitectura tant de programari com d'infraestructura. La meva última tasca ha sigut passar d'una prova de concepte del producte a un MVP i que aquest estigui preparat per escalar, utilitzant arquitectura Domain Driven Design.
		\\ \texttt{Python}\slashsep\texttt{Jira}\slashsep\texttt{Docker}\slashsep\texttt{Linux}\slashsep\texttt{Gitlab Pipelines}\slashsep\texttt{Flask}\slashsep\texttt{OpenAPI}}

	\entry
	{2021 -- 2022\\}
	{Desenvolupador / IT DevOps}
	{UVE Solutions}
	{Empresa del sector del gran consum que proporciona una plataforma per integrar fitxers i
		processar dades connectant així dades de consumidors, distribuïdors i fabricants.

		Vaig començar com a junior al equip d’ETL, la tecnologia utilitzada en tot moment va ser .NET, posteriorment a partir del traspás del cicd a Azure DevOps, i començant a crear microserveis, vaig ser
		assgnat al equip CORE/DevOps on em vaig encarregar de donar suport a IT i de crear microserveis
		i jobs en Kubernetes.
		També he treballat sobre eines com ElasticSearch, RabbitMQ i Prometheus.
		\\ \texttt{C\#}\slashsep\texttt{Jira}\slashsep\texttt{SQL Server}\slashsep\texttt{Windows}\slashsep\texttt{Linux}\slashsep\texttt{Azure DevOps}}
\end{entrylist}
\clearpage
%----------------------------------------------------------------------------------------
%	EDUCATION
%----------------------------------------------------------------------------------------

\cvsect{Educació}

\begin{entrylist}
	\entry
	{2017 -- 2022}
	{Grau Universitari en Enginyeria de Sistemes TIC}
	{Universitat Politècnica de Catalunya}
	{Dins de la meva universitat he fet un munt de tasques en el teixit associatiu, polític i també
		relacionat amb l’Enginyeria. No disposo encara de títol peró tinc justificant confirmant que he superat el 240 ECTS incloent el TFG.}
\end{entrylist}

\cvsect{Experiéncia altres}

\begin{entrylist}
	\entry
	{2019 -- 2021}
	{Coordinador de Política Universitària}
	{Consell de l'Estudiantat.}
	{ Càrrec escollit democràticament pels membres del Consell on vaig desenvolupar tasques de
		coordinació de grups de treball, escriptura de mocions i esmenes, relació entre estudiants
		interuniversitària i propostes a RD o RDL.
		\\ }

	\entry
	{2016 -- 2018\\}
	{Voluntari Creu Roja}
	{Creu Roja Manresa}
	{Durant dos anys vaig estar fent Voluntariat al sector d’infància de Creu Roja on s’aprenien
		dinàmiques d’entreteniment dels infants i també a treballar en equip en altres sectors com Creu
		Roja Juventut. \\ }
\end{entrylist}
%----------------------------------------------------------------------------------------
%	ADDITIONAL INFORMATION
%----------------------------------------------------------------------------------------

\begin{minipage}[t]{0.3\textwidth}
	\vspace{-\baselineskip} % Required for vertically aligning minipages

	\cvsect{Llengues}

	\textbf{Anglès} - Rudimentari\\
	\textbf{Català} - Natiu\\
	\textbf{Castellà} - Natiu
\end{minipage}
\hfill
\begin{minipage}[t]{0.3\textwidth}
	\vspace{-\baselineskip} % Required for vertically aligning minipages

	\cvsect{Aficions}

	M'agraden els videojocs sobre tot el seu desenvolupament, també la ciència i conéixer gent nova.

\end{minipage}
\hfill

%----------------------------------------------------------------------------------------

\end{document}
