%%%%%%%%%%%%%%%%%%%%%%%%%%%%%%%%%%%%%%%%%
% Developer CV
% LaTeX Template
% Version 1.0 (28/1/19)
%
% This template originates from:
% http://www.LaTeXTemplates.com
%
% Authors:
% Jan Vorisek (jan@vorisek.me)
% Based on a template by Jan Küster (info@jankuester.com)
% Modified for LaTeX Templates by Vel (vel@LaTeXTemplates.com)
%
% License:
% The MIT License (see included LICENSE file)
%
%%%%%%%%%%%%%%%%%%%%%%%%%%%%%%%%%%%%%%%%%

%----------------------------------------------------------------------------------------
%	PACKAGES AND OTHER DOCUMENT CONFIGURATIONS
%----------------------------------------------------------------------------------------

\documentclass[9pt]{developercv} % Default font size, values from 8-12pt are recommended

%----------------------------------------------------------------------------------------

\begin{document}

%----------------------------------------------------------------------------------------
%	TITLE AND CONTACT INFORMATION
%----------------------------------------------------------------------------------------

\begin{minipage}[t]{0.6\textwidth} % 45% of the page width for name
	\vspace{-\baselineskip} % Required for vertically aligning minipages
	
	% If your name is very short, use just one of the lines below
	% If your name is very long, reduce the font size or make the minipage wider and reduce the others proportionately
	\colorbox{black}{{\HUGE\textcolor{white}{\textbf{\MakeUppercase{Álvaro}}}}} % First name
	
	\colorbox{black}{{\HUGE\textcolor{white}{\textbf{\MakeUppercase{Mellado}}}}} % Last name
	
	\vspace{6pt}
	
	{\huge Ingeniería de Sistemas TIC} % Career or current job title
\end{minipage}
\begin{minipage}[t]{0.275\textwidth} % 27.5% of the page width for the first row of icons
	\vspace{-\baselineskip}
	\icon{MapMarker}{12}{Manresa}\\
	\icon{Phone}{12}{+34 620 395 887}\\
	\icon{At}{12}{\href{mailto:alvaro@mellado.cat}{alvaro@mellado.cat}}\\	
\end{minipage}
\begin{minipage}[t]{0.275\textwidth} % 27.5% of the page width for the second row of icons
	\vspace{-\baselineskip} % Required for vertically aligning minipages
	
	% The first parameter is the FontAwesome icon name, the second is the box size and the third is the text
	% Other icons can be found by referring to fontawesome.pdf (supplied with the template) and using the word after \fa in the command for the icon you want

\end{minipage}

\vspace{0.5cm}

%----------------------------------------------------------------------------------------
%	INTRODUCTION, SKILLS AND TECHNOLOGIES
%----------------------------------------------------------------------------------------

\cvsect{¿Quién Soy?}

\begin{minipage}[t]{0.4\textwidth} % 40% of the page width for the introduction text
	\vspace{-\baselineskip} % Required for vertically aligning minipages
    Soy un estudiante de Ingeniería de Sistemas TIC, interesado por la soberanía tecnológica, la sostenibilidad de los servicios TIC y las tecnologías web.
    \vspace{\baselineskip}

    Soy una persona inquieta interesada en aportar lo mejor de mí en mi entorno, por eso he participado en la coordinación de equipos en diferentes ámbitos aunque esos no hayan sido laborales.
\end{minipage}
\hfill % Whitespace between
\begin{minipage}[t]{0.5\textwidth} % 50% of the page for the skills bar chart
	\vspace{-\baselineskip} % Required for vertically aligning minipages
	\begin{barchart}{5.5}
		\baritem{Go}{60}
		\baritem{Python}{100}
		\baritem{C}{80}
		\baritem{GIT}{100}
		\baritem{C\#}{80}
	\end{barchart}
\end{minipage}

%----------------------------------------------------------------------------------------
%	EXPERIENCE
%----------------------------------------------------------------------------------------

\cvsect{Experiencia}

\begin{entrylist}
	\entry
	{2024 -- Actualidad\\}
	{Ingeniero de Software}
	{Opportunities Ground SL}
	{ 
		Desarrollo de software para la empresa Opportunities Ground SL, una empresa que se dedica a la creación de software para la gestión de oportunidades de negocio en el sector de los RRHH.
		
		En esta empresa me dedico a la creación de software en Python, utilizando tecnologías como Flask y FastAPI. También me dedico a la creación de pipelines de CI/CD en CircleCI/Github Actions y a la gestión de contenedores Docker.
		\\ \texttt{Python}\slashsep\texttt{Linux}\slashsep\texttt{CircleCI}\slashsep\texttt{Docker}\slashsep\texttt{Flask}\slashsep\texttt{FastAPI}\slashsep\texttt{GitHub Actions}
		}
	\entry
		{2024 -- Actualidad\\}
		{Profesor Asociado}
		{Universitat Politècnica de Catalunya}
		{ 
			Imparto clases de programación en la Universidad Politécnica de Cataluña, en la asignatura de Programación de Sistemas. Enseñando a los alumnos a programar en Python y a entender los conceptos de sistemas operativos tipo UNIX.
		\\ \texttt{Python}\slashsep\texttt{Linux}
		}
		\entry
		{2023 -- 2024}
		{Desarrollador Odoo / DevOps / IT}
		{SOM IT SCCL}
		{Cooperativa de segundo grado que proporciona servicios de consultoría y desarrollo de software a otras cooperativas y empresas.

		Durante este tiempo, me he dedicado a la transformación del stack de infraestructura de la empresa, pasando del clásico stack de Ansible (Inventory, Playbooks, Roles) a Docker i Docker Compose. Todo esto con las singularidades del software de Odoo, que es el software principal de la empresa. También he trabajado en la creación de módulos de Odoo.
		\\ \texttt{Python}\slashsep\texttt{Odoo}\slashsep\texttt{Docker}\slashsep\texttt{Linux}\slashsep\texttt{Gitlab Pipelines}
		}
	\entry
		{2022 -- 2023}
		{Desarrollador backend / DevOps / IT}
		{IZI Record}
		{Empresa emergente del sector multimedia que proporciona un servicio de edición automática de vídeos.

		Trabajé en una empresa donde me dediqué al mantenimiento, gestión y renovación de todo el software no relacionado con Inteligencia Artificial. La herramienta principal fue Flask, pero también trabajé en la renovación de la arquitectura tanto del software como de la infraestructura. Mi última tarea fue pasar de un concepto de prueba del producto a un MVP y prepararlo para escalar, utilizando la arquitectura de Domain Driven Design.
		\\ \texttt{Python}\slashsep\texttt{Jira}\slashsep\texttt{MongoDB}\slashsep\texttt{Linux}\slashsep\texttt{Gitlab Pipelines}\slashsep\texttt{Flask}\slashsep\texttt{OpenAPI}}
	\entry
		{2021 -- 2022}
		{Desarrollador / IT DevOps}
		{UVE Solutions}
		{Empresa del sector de gran consumo que proporciona una plataforma para integrar archivos y procesar datos, conectando así datos de consumidores, distribuidores y fabricantes.

		Comencé como junior en el equipo de ETL, utilizando .NET en todo momento. Posteriormente, con la transición del CI/CD a Azure DevOps y comenzando a crear microservicios, fui asignado al equipo CORE/DevOps, donde me encargué de brindar soporte a IT y crear microservicios y trabajos en Kubernetes. También trabajé con herramientas como ElasticSearch, RabbitMQ y Prometheus.
		\\ \texttt{C\#}\slashsep\texttt{Jira}\slashsep\texttt{SQL Server}\slashsep\texttt{Windows}\slashsep\texttt{Linux}\slashsep\texttt{Azure DevOps}}
\end{entrylist}

%----------------------------------------------------------------------------------------
%	EDUCATION
%----------------------------------------------------------------------------------------
\clearpage


\cvsect{Educación}

\begin{entrylist}
	\entry
		{2017 -- 2021}
		{Grado Universitario en Ingeniería de Sistemas TIC}
		{Universitat Politècnica de Catalunya}
		{Dentro de mi universidad he realizado muchas tareas en el tejido asociativo, político y también relacionado con la Ingeniería. Aún no tengo el título físico, pero tengo un justificante que confirma que he superado los 240 ECTS, incluyendo el TFG.}
\end{entrylist}

\cvsect{Experiencia otros}

\begin{entrylist}
	\entry
	{2019 -- 2021}
	{Coordinador de Política Universitaria}
	{Consell de l'Estudiantat.}
	{Cargo elegido democráticamente por los miembros del Consejo donde desarrollé tareas de coordinación de grupos de trabajo, redacción de mociones y enmiendas, relación entre estudiantes interuniversitarios y propuestas a RD o RDL. \\ }
	\entry
	{2016 -- 2018}
	{Voluntario Cruz Roja}
	{Cruz Roja Manresa}
	{Durante dos años estuve haciendo voluntariado en el sector de la infancia de Cruz Roja, donde aprendí dinámicas de entretenimiento para niños y también a trabajar en equipo en otros sectores como Cruz Roja Juventud. \\ }
\end{entrylist}

%----------------------------------------------------------------------------------------
%	ADDITIONAL INFORMATION
%----------------------------------------------------------------------------------------

\begin{minipage}[t]{0.3\textwidth}
	\vspace{-\baselineskip} % Requerido para alinear verticalmente las minipáginas

	\cvsect{Idiomas}
	
	\textbf{Inglés} - Medio\\
	\textbf{Catalán} - Nativo\\
	\textbf{Español} - Nativo\\
\end{minipage}
\hfill
\begin{minipage}[t]{0.3\textwidth}
	\vspace{-\baselineskip} % Requerido para alinear verticalmente las minipáginas
	
	\cvsect{Aficiones}
	
	Me gustan los videojuegos, especialmente su desarrollo, también la ciencia y conocer gente nueva.
	
\end{minipage}
\hfill

%----------------------------------------------------------------------------------------

\end{document}
