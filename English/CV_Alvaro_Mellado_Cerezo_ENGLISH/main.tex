%%%%%%%%%%%%%%%%%%%%%%%%%%%%%%%%%%%%%%%%%
% Developer CV
% LaTeX Template
% Version 1.0 (28/1/19)
%
% This template originates from:
% http://www.LaTeXTemplates.com
%
% Authors:
% Jan Vorisek (jan@vorisek.me)
% Based on a template by Jan Küster (info@jankuester.com)
% Modified for LaTeX Templates by Vel (vel@LaTeXTemplates.com)
%
% License:
% The MIT License (see included LICENSE file)
%
%%%%%%%%%%%%%%%%%%%%%%%%%%%%%%%%%%%%%%%%%

%----------------------------------------------------------------------------------------
%	PACKAGES AND OTHER DOCUMENT CONFIGURATIONS
%----------------------------------------------------------------------------------------

\documentclass[9pt]{developercv} % Default font size, values from 8-12pt are recommended

%----------------------------------------------------------------------------------------

\begin{document}

%----------------------------------------------------------------------------------------
%	TITLE AND CONTACT INFORMATION
%----------------------------------------------------------------------------------------

\begin{minipage}[t]{0.45\textwidth} % 45% of the page width for name
	\vspace{-\baselineskip} % Required for vertically aligning minipages

	% If your name is very short, use just one of the lines below
	% If your name is very long, reduce the font size or make the minipage wider and reduce the others proportionately
	\colorbox{black}{{\HUGE\textcolor{white}{\textbf{\MakeUppercase{Álvaro}}}}} % First name

	\colorbox{black}{{\HUGE\textcolor{white}{\textbf{\MakeUppercase{Mellado}}}}} % Last name

	\vspace{6pt}

	{\huge Software Engineer} % Career or current job title
\end{minipage}
\begin{minipage}[t]{0.275\textwidth} % 27.5% of the page width for the first row of icons
	\vspace{-\baselineskip} % Required for vertically aligning minipages

	% The first parameter is the FontAwesome icon name, the second is the box size and the third is the text
	% Other icons can be found by referring to fontawesome.pdf (supplied with the template) and using the word after \fa in the command for the icon you want
	\icon{MapMarker}{12}{Manresa}\\
	\icon{Phone}{12}{+34 620 395 887}\\
	\icon{At}{12}{\href{mailto:alvaro@mellado.cat}{alvaro@mellado.cat}}\\
\end{minipage}
\begin{minipage}[t]{0.275\textwidth} % 27.5% of the page width for the second row of icons
	\vspace{-\baselineskip} % Required for vertically aligning minipages

	% The first parameter is the FontAwesome icon name, the second is the box size and the third is the text
	% Other icons can be found by referring to fontawesome.pdf (supplied with the template) and using the word after \fa in the command for the icon you want

\end{minipage}

\vspace{0.5cm}

%----------------------------------------------------------------------------------------
%	INTRODUCTION, SKILLS AND TECHNOLOGIES
%----------------------------------------------------------------------------------------

\cvsect{Who Am I?}

\begin{minipage}[t]{0.4\textwidth} % 40% of the page width for the introduction text
	\vspace{-\baselineskip} % Required for vertically aligning minipages
	I'm a ICT Systems Engineering graduate, interested in technological sovereignty, the sustainability of ICT services and web technologies.\\
	\\
	\\
	I'm a person who is interested in contributing the best of me in my environment, that's why I have participated in the coordination of teams in different areas, even if they have not been work-related.
\end{minipage}
\hfill % Whitespace between
\begin{minipage}[t]{0.5\textwidth} % 50% of the page for the skills bar chart
	\vspace{-\baselineskip} % Required for vertically aligning minipages
	\begin{barchart}{5.5}
		\baritem{C}{80}
		\baritem{Python}{100}
		\baritem{Go}{80}
		\baritem{C\#}{100}
		\baritem{GIT}{100}
	\end{barchart}
\end{minipage}

%----------------------------------------------------------------------------------------
%	EXPERIENCE
%----------------------------------------------------------------------------------------

\cvsect{Experience}

\begin{entrylist}
	\entry
	{2024 -- Present\\}
	{Software Engineer}
	{Opportunities Ground SL}
	{Company specialized in the field of Human Resources and personnel selection. We are responsible for inferring both resumes and job offers to make a match between candidates and companies, saving time in the selection process.
		\\ \texttt{Python}\slashsep\texttt{Linux}\slashsep\texttt{CircleCI}\slashsep\texttt{Docker}\slashsep\texttt{Flask}\slashsep\texttt{FastAPI}\slashsep\texttt{GitHub Actions}}
	\entry
	{2024 -- Present\\}
	{Associate Professor}
	{Universitat Politècnica de Catalunya}
	{Associate professor in the Computer Science subject at EPSEM (Escola Politècnica Superior d'Enginyeria de Manresa) of the UPC. The subject is for first-year students and focuses on programming in Python and the use of Linux.
		\\ \texttt{Python}\slashsep\texttt{Linux}
	}
	\entry
	{2023 -- 2024\\}
	{Odoo Developer / DevOps / IT}
	{SOM IT SCCL}
	{Second Degree Cooperative that provides consulting and software development services to other cooperatives and companies.
		This last months I have been dedicated to the transformation of the company's infrastructure stack, moving from the classic Ansible stack (Inventory, Playbooks, Roles) to Docker and Docker Compose. All this with the singularities of Odoo software, which is the main software of the company. I have also worked on the creation of Odoo modules.
		\\ \texttt{Python}\slashsep\texttt{Odoo}\slashsep\texttt{Docker}\slashsep\texttt{Linux}\slashsep\texttt{Gitlab Pipelines}
	}
	\entry
	{2022 -- 2023\\}
	{Backend developer / DevOps / IT}
	{IZI Record}
	{Emergent company in the multimedia sector that provides an automatic video editing service.
		I worked in it as a backend developer, but also as a DevOps, where I was in charge of the infrastructure and the CI/CD. I also worked on the creation of the OpenAPI documentation and the integration with Jira. My main framework as a python backend developer was Flask. My last task was to move from a product test concept to an MVP and prepare it to scale, using the Domain Driven Design architecture.
		\\ \texttt{Python}\slashsep\texttt{Jira}\slashsep\texttt{MongoDB}\slashsep\texttt{Linux}\slashsep\texttt{Gitlab Pipelines}\slashsep\texttt{Flask}\slashsep\texttt{OpenAPI}
	}
	\entry
	{2021 -- 2022\\}
	{.NET Developer / IT DevOps}
	{UVE Solutions}
	{Company in the fast moving consumer goods sector that provides a platform to integrate files and process data, thus connecting consumer, distributor and manufacturer data.
		I started as a Data Engineer on ETL team, using .NET Framework all the time. Later, with the transition of CI/CD to Azure DevOps and starting to create microservices, I was assigned to the CORE/DevOps team, where I was in charge of providing IT support and creating microservices and jobs in Kubernetes. I also worked with tools like ElasticSearch, RabbitMQ and Prometheus.
		\\ \texttt{C\#}\slashsep\texttt{Jira}\slashsep\texttt{SQL Server}\slashsep\texttt{Windows}\slashsep\texttt{Linux}\slashsep\texttt{Azure DevOps}
	}

\end{entrylist}
\clearpage
%----------------------------------------------------------------------------------------
%	EDUCATION
%----------------------------------------------------------------------------------------

\cvsect{Education}

\begin{entrylist}
	\entry
	{2017 -- 2021}
	{Bachelor's Degree in ICT Systems Engineering}
	{Universitat Politècnica de Catalunya- UPC}
	{In my university I have done many tasks in the associative, political and also related to Engineering. I still do not have the physical title yet, but I have a certificate that confirms that I have passed the 240 ECTS, including the Bachelor's Thesis.\\
		\\ \\ \\}
\end{entrylist}



\cvsect{Other Experiences}

\begin{entrylist}
	\entry
	{2019 -- 2021}
	{President of the Student Council}
	{Student Council of the UPC.}
	{Position democratically chosen by the members of the Council where I carried out tasks such as coordinating work groups, drafting motions and amendments, fostering relationships among interuniversity students, and making proposals to some goverment rules.\\
	}
	\entry
	{2016 -- 2018\\}
	{Red Cross Volunteer}
	{Red Cross Manresa}
	{During two years I was doing volunteer work in the children's sector of the Red Cross, where I learned entertainment dynamics for children and also to work as a team in other sectors such as Red Cross Youth.\\
	}
\end{entrylist}

%----------------------------------------------------------------------------------------
%	ADDITIONAL INFORMATION
%----------------------------------------------------------------------------------------

\begin{minipage}[t]{0.3\textwidth}
	\vspace{-\baselineskip} % Requerido para alinear verticalmente las minipáginas

	\cvsect{Languages}

	\textbf{Inglés} - Conversational\\
	\textbf{Catalán} - Native\\
	\textbf{Español} - Native
\end{minipage}
\hfill
\begin{minipage}[t]{0.3\textwidth}
	\vspace{-\baselineskip} % Requerido para alinear verticalmente las minipáginas

	\cvsect{Hobbies}
	\\
	I like video games, especially their development, also science and meeting new people.

\end{minipage}
\hfill

%----------------------------------------------------------------------------------------

\end{document}
